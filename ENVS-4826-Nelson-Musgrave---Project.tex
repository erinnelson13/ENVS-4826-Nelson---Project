% Options for packages loaded elsewhere
\PassOptionsToPackage{unicode}{hyperref}
\PassOptionsToPackage{hyphens}{url}
%
\documentclass[
]{article}
\usepackage{amsmath,amssymb}
\usepackage{lmodern}
\usepackage{ifxetex,ifluatex}
\ifnum 0\ifxetex 1\fi\ifluatex 1\fi=0 % if pdftex
  \usepackage[T1]{fontenc}
  \usepackage[utf8]{inputenc}
  \usepackage{textcomp} % provide euro and other symbols
\else % if luatex or xetex
  \usepackage{unicode-math}
  \defaultfontfeatures{Scale=MatchLowercase}
  \defaultfontfeatures[\rmfamily]{Ligatures=TeX,Scale=1}
\fi
% Use upquote if available, for straight quotes in verbatim environments
\IfFileExists{upquote.sty}{\usepackage{upquote}}{}
\IfFileExists{microtype.sty}{% use microtype if available
  \usepackage[]{microtype}
  \UseMicrotypeSet[protrusion]{basicmath} % disable protrusion for tt fonts
}{}
\makeatletter
\@ifundefined{KOMAClassName}{% if non-KOMA class
  \IfFileExists{parskip.sty}{%
    \usepackage{parskip}
  }{% else
    \setlength{\parindent}{0pt}
    \setlength{\parskip}{6pt plus 2pt minus 1pt}}
}{% if KOMA class
  \KOMAoptions{parskip=half}}
\makeatother
\usepackage{xcolor}
\IfFileExists{xurl.sty}{\usepackage{xurl}}{} % add URL line breaks if available
\IfFileExists{bookmark.sty}{\usepackage{bookmark}}{\usepackage{hyperref}}
\hypersetup{
  pdftitle={ENVS 4826 Nelson Musgrave - Project},
  pdfauthor={Erin Nelson},
  hidelinks,
  pdfcreator={LaTeX via pandoc}}
\urlstyle{same} % disable monospaced font for URLs
\usepackage[margin=1in]{geometry}
\usepackage{graphicx}
\makeatletter
\def\maxwidth{\ifdim\Gin@nat@width>\linewidth\linewidth\else\Gin@nat@width\fi}
\def\maxheight{\ifdim\Gin@nat@height>\textheight\textheight\else\Gin@nat@height\fi}
\makeatother
% Scale images if necessary, so that they will not overflow the page
% margins by default, and it is still possible to overwrite the defaults
% using explicit options in \includegraphics[width, height, ...]{}
\setkeys{Gin}{width=\maxwidth,height=\maxheight,keepaspectratio}
% Set default figure placement to htbp
\makeatletter
\def\fps@figure{htbp}
\makeatother
\setlength{\emergencystretch}{3em} % prevent overfull lines
\providecommand{\tightlist}{%
  \setlength{\itemsep}{0pt}\setlength{\parskip}{0pt}}
\setcounter{secnumdepth}{-\maxdimen} % remove section numbering
\ifluatex
  \usepackage{selnolig}  % disable illegal ligatures
\fi

\title{ENVS 4826 Nelson Musgrave - Project}
\author{Erin Nelson}
\date{02/11/2021}

\begin{document}
\maketitle

\hypertarget{introduction}{%
\subsection{Introduction}\label{introduction}}

Trees and other vegetation provide a wide range of benefits to urban
settings. Some of these benefits include cooling of the environment,
absorbing stormwater, enhancing aesthetics and notably, absorbing carbon
dioxide and pollutants (Duinker et al., 2015). One area in Halifax that
has an abundance of trees for these reasons is the Halifax Waterfront.
The waterfront area that was analysed has over 300 trees with a wide
range of species and ages. The trees found at the Halifax Waterfront are
managed by DevelopNS.

While it is possible that a few of these trees grew naturally, and were
developed around, the majority of these trees have been planted and
maintained over the course of their life. While it's known that trees
are beneficial for reducing pollution in urban settings (Guidolotti et
al., 2016) , our question is whether pollution has a negative impact on
tree health and growth. More specifically, we are going to analyse
whether trees in closer proximity to the road have a negative impact on
their health. The World Health Organization listed road pollutants as
one major source of air pollutants globally, affecting both humans and
local ecological systems (Aggarwal \& Jain, 2015). While pollutants are
a concern for trees near roads there is also the danger of physical
damage to trees due to high traffic. We will look at these issues by
comparing trees near roads to those that are not. We hypothesize that
the trees near roads will be in less healthy conditions with respect to
crown coverage, trunk damage, and leaf health.

\hypertarget{methods}{%
\subsection{Methods}\label{methods}}

\end{document}
